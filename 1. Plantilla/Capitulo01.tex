%%%%%%%%%%%%%%%%%%%%%%%%%%%%%%%%%%%%%%%%%%%%%%%%%%%%%%%%%%%%%%%%%%%
%%% Documento LaTeX 																						%%%
%%%%%%%%%%%%%%%%%%%%%%%%%%%%%%%%%%%%%%%%%%%%%%%%%%%%%%%%%%%%%%%%%%%
% Título:		Introducción
% Autor:  	Ignacio Moreno Doblas
% Fecha:  	2014-02-01, actualizado 2019-11-11
% Versión:	0.5.0
%%%%%%%%%%%%%%%%%%%%%%%%%%%%%%%%%%%%%%%%%%%%%%%%%%%%%%%%%%%%%%%%%%%
% !TEX root = A0.MiTFG.tex

\chapterbegin{Introducción y visión general}
\minitoc

\section{Problemática de los accidentes de tráfico}
Poner lo que hice para la introducción en proyectos y sistemas. Hablar sobre la necesidad de la aplicación de VVLC para mejorar la seguridad vial.

\section{Problemática de las ondas electromagnéticas}
Poner lo que hice para la introducción en proyectos y sistemas. Sobre todo para contextualizar por qué es mas conveniente no usar esta comunicación.

\section{Estado del arte}

Historia de las VLC y de las VVLC, comentar cuando y como surgió esta idea y hasta donde se ha desarrollado actualmente

Tambien tener un apartado para hablar sobre los proyectos de VVLC que se han tomado de referencia para entender y desarrollar este proyecto. Introduccion proyectos y sistemas. 

\section{Esquemas de señalización}

Hablar sobre la importancia de los esquemas de señalización y de que son el enfoque principal del TFG.

\section{Estructura del documento}

En esta sección, se explican los posteriores capítulos u otra información adicional que el proyecto contenga.

\section{Ámbito de aplicación}

Desarrollar que el principal ámbito de aplicación del proyecto es en la comunicación vehicular pero que puede tener importancia en otras aplicaciones (cualquier tipo de comunicación)

\section{Objetivo}
\label{sec:intro:obj}
El objetivo global de este proyecto es la realización de un sistema de comunicación por luz
visible orientado a vehículos a través de una matriz de puertas lógicas programable
(FPGA) que actúa de intermediaria entre el transmisor y el receptor, cumpliendo con el
estándar IEE 802.15.7-218.
Este objetivo se desarrollará implementando varias técnicas de transmisión y recepción siendo las más destacables la implementación de 
diferentes esquemas de codificación de la señal desarrollando el codificador y el decodificador y la implementación de distintos sistemas de 
decisión para interpretar la señal recibida antes de decodificarla. 
Además, el sistema deberá tener robustez ante posibles
efectos adversos provocados por las condiciones meteorológicas y un rango de
alcance lo suficientemente alto para cubrir una distancia considerable entre coches.

\chapterend

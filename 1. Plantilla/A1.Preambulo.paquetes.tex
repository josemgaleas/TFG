%%%%%%%%%%%%%%%%%%%%%%%%%%%%%%%%%%%%%%%%%%%%%%%%%%%%%%%%%%%%%%%%%%%
%%% Documento LaTeX 																						%%%
%%%%%%%%%%%%%%%%%%%%%%%%%%%%%%%%%%%%%%%%%%%%%%%%%%%%%%%%%%%%%%%%%%%
% Título:	Paquetes
% Autor:  Ignacio Moreno Doblas
% Fecha:  2014-02-01, actualizado 2019-11-11
%%%%%%%%%%%%%%%%%%%%%%%%%%%%%%%%%%%%%%%%%%%%%%%%%%%%%%%%%%%%%%%%%%%
% Tabla de materias:
%	1 Codificación e idioma %
% 2 Matemáticas y Física %
% 3 Gráficos%
% 4 Estilo y formato%
%%%%%%%%%%%%%%%%%%%%%%%%%%%%%%%%%%%%%%%%%%%%%%%%%%%%%%%%%%%%%%%%%%%

%1 Codificación e idioma%
\usepackage[utf8]{inputenc} %Codificación en latin-1%
\usepackage[T1]{fontenc} %Codificación de fuente%
\usepackage[spanish]{babel}	%Hyphenation (Guionado) en español%
\usepackage{eurosym} %Tipografía euro (€)%

%2 Matemáticas y Física %
% Importante para ecuaciones, magnitudes y unidades%
\usepackage{amssymb,amsmath,latexsym,amsfonts} % paquetes estándar%
\usepackage[squaren]{SIunits} %Paquete para magnitudes y unidades físicas%
\usepackage{ifthen} %sentencias if y while%

%3 Gráficos%
\usepackage{graphics,graphicx,subcaption} %paquetes gráficos estándar%
%\usepackage{subfigure} %permite varias subfiguras en una figura
\usepackage{wrapfig} %paquete para gráfica lateral%
\usepackage{float} %figuras flotantes%
	% \begin{floatingfigure}[r]/[l]{4.5cm}
	% \end{floatingfigure}
\usepackage{graphpap}	%comando \graphpaper en el entorno picture%

%4 Estilo y formato%
\usepackage{imakeidx} %MakeIndex%
%Comando para crear glosario (index en inglés)
\makeindex[columns=2, title=Glosario, intoc]
%\usepackage{showidx} % Hace que cada comando \index se imprima en la página donde se ha puesto (útil para corregir los índices)
\usepackage{fancyhdr}	%cabeceras y pies mejor que con \pagestyle{}%
\setlength{\headheight}{16pt}% ...at least 16pt
\usepackage{titlesec,titletoc} %Formateo de secciones y títulos%
\raggedbottom %Para fragmentar versos en varias páginas%
\usepackage{alltt} % Define el environment alltt, como verbatim, excepto que \, { y } tienen su significado normal. Se describe en el fichero alltt.dtx.
\usepackage[pdftex,bookmarksnumbered,hidelinks]{hyperref} %hyper-references%
\usepackage{minitoc} % Para poner tablas de contenido en cada capítulo.
\usepackage{listings} % Para escribir piezas de código C, Python, etc. %
%listings configuration
\lstset{
  language=Python, %Puede ser C, C++, Java, etc.
  showstringspaces=false,
  formfeed=\newpage,
  tabsize=4,
  commentstyle=\itshape,
  basicstyle=\footnotesize\ttfamily,
  escapeinside={(@}{@)},
  morekeywords={models, lambda, forms}
}
\renewcommand{\lstlistingname}{Listado}

\usepackage{tipa} % tipografía IPA (International Phonetic Alphabet)
\usepackage{longtable} %Entorno Longtable, fracciona tablas a lo largo de páginas%
\usepackage{xcolor}
\usepackage{colortbl}
\usepackage{acronym}  %Para expandir automáticamente los acrónimos
\usepackage{emptypage} %Para eliminar el número de página en páginas sin contenido (\cleardoublepage)

\usepackage{tikz}



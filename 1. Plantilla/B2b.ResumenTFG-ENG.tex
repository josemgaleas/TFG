%%%%%%%%%%%%%%%%%%%%%%%%%%%%%%%%%%%%%%%%%%%
% Página de resumen del proyecto en inglés%
%%%%%%%%%%%%%%%%%%%%%%%%%%%%%%%%%%%%%%%%%%%
% !TEX root = A0.MiTFG.tex

\pagestyle{fancy}
\addstarredchapter{Abstract}

\begin{center}
	\scshape
	E.T.S. de Ingeniería de Telecomunicación, Universidad de Málaga
\end{center}

\bigskip

\begin{center}
	%\Large 
	\scshape
	\textbf{Development of a VVLC communications system with signaling schemes in SoC FPGA}
\end{center}

\bigskip \bigskip \bigskip

\begin{minipage}{\textwidth}

\textbf{Author:} José Miguel Galeas Merchán

\medskip

\textbf{Supervisor:} Antonio García Zambrana

% \medskip

% \textbf{Co-supervisor:} <Nombre del cotutor>\ (elimina esta línea si no hay cotutor)

\medskip

\textbf{Department:} Ingeniería de Comunicaciones

\medskip

\textbf{Degree:} Grado en Ingeniería de Sistemas Electrónicos

\medskip

\textbf{Keywords:} FPGA, signaling schemes, decision systems, visible light,
Red Pitaya, distance, security, communications. 
%Keywords (separated by commas) describing and characterizing the topic of the work.

\bigskip \bigskip


\end{minipage}

\begin{center}
	\textbf{Abstract}
\end{center}

% The abstract should briefly describe the project context, goals and
% obtained results. It should not exceed this page.

Currently, remote communications are part of our daily lives and,
undoubtedly, visible light communications are already a present. In addition, they are
vital for improve the safety and protection of people on the road and to decrease the use
of electromagnetic waves. That is why the objective of this project is to realize a 
vehicular visible light communication system implementing different signaling schemes and 
decision systems in a SoC FPGA to provide robustness to the system and to be able to 
carry out communications over long distances.

In the first place, a theorical study of the signaling schemes and of the decision systems
implemented has been done to describe the bases on which the job has been built.

Second, the design of the system and its implementation in the Soc
FPGA using the Red Pitaya development board and the programming environment
Vivado under the VHDL language have been detailed.

Lastly, different test have been done to verify the functionality of the system.
The results obtained have been very satisfactory improving considerably the starting
communication system.

\blankpage

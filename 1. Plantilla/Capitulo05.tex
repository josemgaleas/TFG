%%%%%%%%%%%%%%%%%%%%%%%%%%%%%%%%%%%%%%%%%%%%%%%%%%%%%%%%%%%%%%%%%%%
%%% Documento LaTeX 																						%%%
%%%%%%%%%%%%%%%%%%%%%%%%%%%%%%%%%%%%%%%%%%%%%%%%%%%%%%%%%%%%%%%%%%%
% Título:		Capítulo 5
% Autor:  	Ignacio Moreno Doblas
% Fecha:  	2014-02-01, actualizado 2019-11-11
% Versión:	0.5.0
%%%%%%%%%%%%%%%%%%%%%%%%%%%%%%%%%%%%%%%%%%%%%%%%%%%%%%%%%%%%%%%%%%%
% !TEX root = A0.MiTFG.tex

\chapterbegin{Implementación}
\label{chp:App2}
\minitoc

\section{Sistema general}
Comentar como se relacionan los bloques entre sí y el orden de los mismos para luego entrar en profundidad en 
los bloques que se han desarrollado.

\section{Transmisor}
		-Selector del esquema de codificación (sincronismo manchester y trama en la codificación elegida)
		-Diagramas de Markov para implementar la codificación

\section{Receptor}
		-Recepción del dato serie
		-Agrupar los datos en parejas o cuartetos
		-Decodificar con máquinas de estados o identificación del bit mayor.

\section{Sistema de decisión} (hard-soft decoding, algoritmo de viterbi)
		-Hard-decoding (umbral en la mitad)
		-Soft-decoding (cómo se implementa el cálculo de la distancia euclídea)
		-Algoritmo de Viterbi (que efecto tiene y cómo se implementa la mirada al pasado para descartar opciones que tienen probabilidad 0)
\chapterend{}
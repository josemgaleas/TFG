% !TEX root = A0.MiTFG.tex

\chapterbeginx{Conclusiones y líneas futuras}

% Después de todo el desarrollo del proyecto, es pertinente hacer una
% valoración final del mismo, respecto a los resultados obtenidos, las
% expectativas o el resultado de la experiencia acumulada.

% Esta sección es indispensable y en ella se ha de reflejar, lo más
% claramente posible, las aportaciones del trabajo con unas conclusiones
% finales.

% Además, considerando también el estado de la técnica, se deben indicar
% las posibles líneas futuras de trabajo, proponer otros puntos de vista
% o cualquier otra sugerencia como postámbulo del presente trabajo, para
% ser considerada por el lector o el tribunal evaluador.

En este último capítulo se van a exponer algunas conclusiones y líneas futuras que se 
han pensado que se podrían implementar después de la realización del proyecto.

Tras la realización de este trabajo se ha comprendido en profundidad los fundamentos 
de los esquemas de señalización para comunicaciones digitales además del gran impacto
que supone la inclusión de los sistemas de decisión a la hora de interpretar y 
decodificar la señal. A su vez, se han aumentado las capacidades y conocimientos a la 
hora de trabajar con dispositivos FPGA tanto a nivel hardware como software siendo 
las más relevantes el conocimiento de la estructura interna de estos dispositivos y
su entorno de programación. También se ha mejorado en el uso y en la comprensión de 
los dispositivos conversores y de los transceptores ópticos para realizar comunicaciones
ópticas.

Sin duda, el mayor reto ha sido la programación de la FPGA ya 
que se ha tenido que cambiar la visión de programación a la que nos habituamos durante 
el Grado porque para programar la FPGA se utiliza
un lenguaje de descripción hardware, en concreto, VHDL. 

Como conclusión, se puede decir que el trabajo realizado ha sido satisfactorio con el 
objetivo planteado siendo este dotar de mayor robustez al sistema mediante la implementación
de esquemas de codificación junto con varios sistemas de decisión. A decir verdad, los 
resultados han superado las espectativas planteadas ya que se han conseguido realizar
comunicaciones a 60 metros de distancia cuando en el sistema inicial era de 2-3 metros por
lo que puede ser integrado en vehículos para ser testeado.

\newpage
Hay diversas líneas futuras que se pueden incorporar al sistema para completar 
y mejorar su funcionalidad. Estas se van a comentar a continuación:

\begin{itemize}
    \item Diseñar e implementar otros esquemas de señalización ampliando el área a 
    modulaciones en frecuencia, de fase o de amplitud.
    \item Estudiar la rentabilidad de aumentar el algoritmo de Viterbi a costa de la 
    complejidad e intentar disminuir los retrasos en los cálculos de las distancias 
    euclídeas. Además, investigar si es ventajoso cambiar la versión de la Red Pitaya por 
    otra más potente.
    \item Diseñar un filtro paso alto a la entrada
    ya que una vez recibida la señal es conveniente
    calcular que filtro paso alto es el adecuado para mejorar la calidad de
    la señal recibida.
    \item Desarrollar un interfaz para visualizar la información descrita de una manera
    más representativa e intuitiva lo que puede ser ventajoso a la hora de implementarlo
    en un vehículo.
    \item Implementar una aplicación móvil que permita la transmisión y recepción de datos
    ya que hoy en día, el uso de dispositivos móviles se ha incrementado mucho, 
    tanto que casi toda la población usa uno, por lo que sería una gran ventaja para 
    aplicaciones futuras.
\end{itemize}

\chapterend

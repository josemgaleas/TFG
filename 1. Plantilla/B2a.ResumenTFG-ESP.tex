%%%%%%%%%%%%%%%%%%%%%%%%%%%%%%%%%%
% Página de resumen del proyecto %
%%%%%%%%%%%%%%%%%%%%%%%%%%%%%%%%%%
% !TEX root = A0.MiTFG.tex

\pagestyle{fancy}
\renewcommand{\headrulewidth}{0pt}
\addstarredchapter{Resumen}

\begin{center}
	\scshape
	E.T.S. de Ingeniería de Telecomunicación, Universidad de Málaga
\end{center}

\bigskip

\begin{center}
	%\Large 
	\scshape
	\textbf{Desarrollo de un sistema de comunicaciones VVLC con implementación de esquemas de 
	señalización en SoC FPGA}
\end{center}

\bigskip \bigskip \bigskip

\begin{minipage}{\textwidth}

\textbf{Autor:}  José Miguel Galeas Merchán

\medskip

\textbf{Tutor:} Antonio García Zambrana

%\medskip

%\textbf{Cotutor:} <Nombre del cotutor>\ (elimina esta línea si no hay cotutor)

\medskip

\textbf{Departamento:} Ingeniería de Comunicaciones

\medskip

\textbf{Titulación:} Grado en Ingeniería de Sistemas Electrónicos

\medskip

\textbf{Palabras clave:} FPGA, esquemas de señalización, sistemas de decisión, 
luz visible, Red Pitaya, distancia, seguridad, comunicaciones.
%Palabras clave (separadas por coma) que describen y caracterizan el tema del trabajo.

\bigskip \bigskip


\end{minipage}

\begin{center}
	\textbf{Resumen}
\end{center}

% El resumen debe ser una breve descripción del contexto del proyecto,
% sus objetivos y los resultados obtenidos. Se recomienda que no exceda
% esta página.
Actualmente, las comunicaciones a distancia forman parte de nuestra vida cotidiana y, sin 
duda, las comunicaciones por luz visible ya son un presente. Además, son vitales para 
mejorar la seguridad y protección de las personas en la carretera y para disminuir el uso 
de ondas electromagnéticas. Es por ello, que el objetivo
de este proyecto es la realización de un sistema de comunicación por luz visible 
orientado a vehículos
implementando distintos esquemas de señalización y de decisión en un Soc FPGA para dotar de 
robustez al sistema y poder realizar comunicaciones a grandes distancias.

En primer lugar, se ha realizado un estudio teórico de los esquemas de señalización 
implementados y de los sistemas de decisión para describir las bases sobre las cuales se 
ha construido el trabajo.

En segundo lugar, se ha detallado el diseño del sistema y su implementación en el Soc 
FPGA utilizando para ello la placa de desarrollo Red Pitaya y el entorno de programación
Vivado bajo el lenguaje VHDL.

En último lugar, se han realizado diferentes pruebas para comprobar el funcionamiento del 
sistema. Los resultados obtenidos han sido muy satisfactorios 
mejorando considerablemente la distancia de comunicación de partida.


\blankpage

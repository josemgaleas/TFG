%%%%%%%%%%%%%%%%%%%%%%%%%%%%%%%%%%%%%%%%%%%%%%%%%%%%%%%%%%%%%%%%%%%
%%% Documento LaTeX 																						%%%
%%%%%%%%%%%%%%%%%%%%%%%%%%%%%%%%%%%%%%%%%%%%%%%%%%%%%%%%%%%%%%%%%%%
% Título:		Capítulo 3
% Autor:  	Ignacio Moreno Doblas
% Fecha:  	2014-02-01, actualizado 2019-11-11
% Versión:	0.5.0
%%%%%%%%%%%%%%%%%%%%%%%%%%%%%%%%%%%%%%%%%%%%%%%%%%%%%%%%%%%%%%%%%%%
% !TEX root = A0.MiTFG.tex

\chapterbegin{Fundamentos teóricos}
\label{chp:App}
\minitoc

\section{Introducción a los esquemas de señalización}
%Hablar sobre qué són los esquemas de señalización y cual es su función principal. comentar
%los esquemas (alternos,cancelación,4ppm,4ippm y pwm) y por qué se han elegido o descartado.
%Decir que no son los esquemas del estándar y que se han elegido para investigar sus ventajas e inconvenientes. 

En el ámbito de la comunicación existen múltiples esquemas de codificación digital con diferentes propiedades como 
probabilidad de bit, ciclo de trabajo, ancho de banda, etc. A la hora de estudiar un esquema de codificación para hacer 
su elección hay que tener en cuenta tres aspectos fundamentales, que son:
\begin{itemize}
    \item Flickering: 
    \item Rendimiento óptico
    \item La capacidad para controlar la atenuación o el dimming, 
\end{itemize} 

El estándar de comunicaciones por luz visible IEEE 802.15.7 usa como esquema de señalización la codificación Manchester. 
Continuando con el estudio de los esquemas de señalización,
a continuación, se van a desarrollar otras opciones de esquemas de codificación con características diferentes
para estudiar su eficacia e impacto en las comunicaciones por luz visible.

\section{Pulsos alternos}
Comentar teóricamente en que consiste este esquema (miller pero con una opción menos y alternando).
Y que apenas está implementado y desarrollado en ningún sitio.

Dibujar y comentar el diagrama de trellis de este esquema para la codificación/decodificación de la señal.

si encuentro algo hablar sobre cual es su espectro y las diferencias respecto a los del estándar

Desarrollar por qué es un esquema no equiprobable haciendo los cálculos correspondientes.

Comentar si se encuentra la ber y sus prestaciones

\section{Pulsos alternos con cancelación de pulsos}
Comentar teóricamente en que consiste este esquema (alternos pero eliminando pulsos).
Y que apenas está implementado y desarrollado en ningún sitio.

Dibujar y comentar el diagrama de trellis de este esquema para la codificación/decodificación de la señal.

si encuentro algo hablar sobre cual es su espectro y las diferencias respecto a los del estándar

Desarrollar por qué es un esquema no equiprobable haciendo los cálculos correspondientes.

Comentar si se encuentra la ber y sus prestaciones

\section{4-PPM}
Explayarme mucho más porque hay mucha más información.
Comentar teóricamente en que consiste este esquema.
Y que apenas está implementado y desarrollado en ningún sitio.

Dibujar y comentar el diagrama de trellis de este esquema para la codificación/decodificación de la señal.

si encuentro algo hablar sobre cual es su espectro y las diferencias respecto a los del estándar

Desarrollar por qué es un esquema no equiprobable haciendo los cálculos correspondientes.

Comentar si se encuentra la ber y sus prestaciones

\section{Comparativa 4-PPM frente a Inverse 4-PPM}
Comparar estos dos esquemas y comentar ventajas e inconvenientes.
%https://ieeexplore.ieee.org/stamp/stamp.jsp?tp=&arnumber=6614862
%https://ieeexplore.ieee.org/stamp/stamp.jsp?tp=&arnumber=567560&tag=1

\section{Sistemas de decisión}
Hablar sobre las técnicas usadas --> hard-decoding, soft-decoding y algoritmo de Viterbi
Tener en cuenta el mapeo que se realiza para poner la señal en el rango [-8192,8191]

\chapterend{}

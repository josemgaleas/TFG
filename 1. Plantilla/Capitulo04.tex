%%%%%%%%%%%%%%%%%%%%%%%%%%%%%%%%%%%%%%%%%%%%%%%%%%%%%%%%%%%%%%%%%%%
%%% Documento LaTeX 																						%%%
%%%%%%%%%%%%%%%%%%%%%%%%%%%%%%%%%%%%%%%%%%%%%%%%%%%%%%%%%%%%%%%%%%%
% Título:		Capítulo 3
% Autor:  	Ignacio Moreno Doblas
% Fecha:  	2014-02-01, actualizado 2019-11-11
% Versión:	0.5.0
%%%%%%%%%%%%%%%%%%%%%%%%%%%%%%%%%%%%%%%%%%%%%%%%%%%%%%%%%%%%%%%%%%%
% !TEX root = A0.MiTFG.tex

\chapterbegin{Fundamentos teóricos}
\label{chp:App}
\minitoc

\section{Introducción a los esquemas de señalización}
%Hablar sobre qué són los esquemas de señalización y cual es su función principal. comentar
%los esquemas (alternos,cancelación,4ppm,4ippm y pwm) y por qué se han elegido o descartado.
%Decir que no son los esquemas del estándar y que se han elegido para investigar sus ventajas e inconvenientes. 
Un esquema de codificación estandariza la codificación de caracteres mediante la definición de un 
método único para representar los datos de tipo carácter. 

Para los sistemas de comunicación por luz visible el estándar define el código Manchester como el sistema de codificación a 
usar. Sin embargo, existen muchas otras alternativas a este código que se van a describir a continuación.

\section{Pulsos alternos}
Comentar teóricamente en que consiste este esquema (miller pero con una opción menos y alternando).
Y que apenas está implementado y desarrollado en ningún sitio.

Dibujar y comentar el diagrama de trellis de este esquema para la codificación/decodificación de la señal.

si encuentro algo hablar sobre cual es su espectro y las diferencias respecto a los del estándar

Desarrollar por qué es un esquema no equiprobable haciendo los cálculos correspondientes.

Comentar si se encuentra la ber y sus prestaciones

\section{Pulsos alternos con cancelación de pulsos}
Comentar teóricamente en que consiste este esquema (alternos pero eliminando pulsos).
Y que apenas está implementado y desarrollado en ningún sitio.

Dibujar y comentar el diagrama de trellis de este esquema para la codificación/decodificación de la señal.

si encuentro algo hablar sobre cual es su espectro y las diferencias respecto a los del estándar

Desarrollar por qué es un esquema no equiprobable haciendo los cálculos correspondientes.

Comentar si se encuentra la ber y sus prestaciones

\section{4-PPM}
Explayarme mucho más porque hay mucha más información.
Comentar teóricamente en que consiste este esquema.
Y que apenas está implementado y desarrollado en ningún sitio.

Dibujar y comentar el diagrama de trellis de este esquema para la codificación/decodificación de la señal.

si encuentro algo hablar sobre cual es su espectro y las diferencias respecto a los del estándar

Desarrollar por qué es un esquema no equiprobable haciendo los cálculos correspondientes.

Comentar si se encuentra la ber y sus prestaciones

\section{Comparativa 4-PPM frente a Inverse 4-PPM}
Comparar estos dos esquemas y comentar ventajas e inconvenientes.
%https://ieeexplore.ieee.org/stamp/stamp.jsp?tp=&arnumber=6614862
%https://ieeexplore.ieee.org/stamp/stamp.jsp?tp=&arnumber=567560&tag=1

\section{Sistemas de decisión}
Hablar sobre las técnicas usadas --> hard-decoding, soft-decoding y algoritmo de Viterbi
Tener en cuenta el mapeo que se realiza para poner la señal en el rango [-8192,8191]

\chapterend{}

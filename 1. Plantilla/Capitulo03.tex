%%%%%%%%%%%%%%%%%%%%%%%%%%%%%%%%%%%%%%%%%%%%%%%%%%%%%%%%%%%%%%%%%%%
%%% Documento LaTeX 																						%%%
%%%%%%%%%%%%%%%%%%%%%%%%%%%%%%%%%%%%%%%%%%%%%%%%%%%%%%%%%%%%%%%%%%%
% Título:		Capítulo 2
% Autor:  	Ignacio Moreno Doblas
% Fecha:  	2014-02-01, actualizado 2019-11-11
% Versión:	0.5.0
%%%%%%%%%%%%%%%%%%%%%%%%%%%%%%%%%%%%%%%%%%%%%%%%%%%%%%%%%%%%%%%%%%%
% !TEX root = A0.MiTFG.tex

\chapterbegin{Sistema de comunicación}
\label{chp:Utiliz}
\minitoc

\section{Estándar de los sistemas VLC}
%\label{sec:TestSuiteCrec}
La entidad que realiza el estándar es IEEE 802 que realizó el primer estándar oficial de
comunicación por luz visible en 2011 [802.15.7-2011]. Este estándar fue revisado en
2018 y se publicó una segunda versión que es 802.15.7-2018 IEEE Standard for Local
and metropolitan área networks. Part 15.7: Short-Range Optical Wireless.
[DESARROLLAR MÁS EL ESTÁNDAR]

\section{Enlace de luz}
Para probar el proyecto se parte de un enlace de luz formado por un transmisor y un receptor.

\subsection{Transmisor}
Hablar un poco del transmisor óptico que tiene Salva y de los voltajes.

\subsection{Receptor}
Hablar un poco del receptor óptico que tiene Salva y de los voltajes.

\section{Sistema de partida}
Hablar sobre el proyecto que nos dejó Andrés tanto de las características
software como su capacidad de alcance y el enfoque del trabajo de Andrés que era el Desarrollo
del filtro adaptado y al final tratar de hilarlo con nuestra implementación para mejorar el sistema.
Comentar el mapeo de la señal para ponerla en el rango [-8192,8191] que nos influye para el hard-decoding y el soft-decoding.

\section{Mejoras respecto al sistema anterior}
Hablar sobre cuales son las mejoras que se han implementado y que es lo que se quiere conseguir con estas mejoras.
Las mejoras (Mayor robustez de paquetes y mayor distancia de transmisión) se han conseguido
gracias a implementar esquemas de codificación y sistemas de decisión.

\chapterend{}

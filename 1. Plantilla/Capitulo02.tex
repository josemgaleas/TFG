%%%%%%%%%%%%%%%%%%%%%%%%%%%%%%%%%%%%%%%%%%%%%%%%%%%%%%%%%%%%%%%%%%%
%%% Documento LaTeX 																						%%%
%%%%%%%%%%%%%%%%%%%%%%%%%%%%%%%%%%%%%%%%%%%%%%%%%%%%%%%%%%%%%%%%%%%
% Título:		Capítulo 1
% Autor:  	Ignacio Moreno Doblas
% Fecha:  	2014-02-01, actualizado 2019-11-11
% Versión:	0.5.0
%%%%%%%%%%%%%%%%%%%%%%%%%%%%%%%%%%%%%%%%%%%%%%%%%%%%%%%%%%%%%%%%%%%
% !TEX root = A0.MiTFG.tex
\chapterbegin{Tecnología empleada}
En este apartado se van a explicar las plataformas o elementos, tanto hardware como 
software, sobre los que se ha desarrollado este proyecto. 
\vspace{1cm}
\label{chp:ManLaTeX}
\minitoc

\section{Plataforma hardware}
El sistema hardware empleado va a ser un dispositivo FPGA. Una matriz de puertas lógicas 
programables (FPGA) se define como un dispositivo electrónico programable que contiene 
bloques de lógica cuya interconexión y funcionalidad puede ser configurada en el momento, 
mediante un lenguaje de descripción especializado. La lógica programable puede reproducir 
desde funciones tan sencillas como las llevadas a cabo por una puerta lógica o un sistema 
combinacional hasta complejos sistemas en un chip. Su principal 
ventaja es que pueden ser reprogramados para un trabajo específico o cambiar sus 
requisitos después de haberse fabricado. El inventor de esta tecnología fue 
Xilinx. 

La principal característica es la flexibilidad. Esto también implica que en muchos
casos se pueden hacer cambios físicos sin hacer modificaciones costosas en la placa.

La segunda característica es la aceleración. En el proceso de producción, 
estos dispositivos son muy fáciles de fabricar y se venden preparados para ser usados 
directamente, lo cual convella una disminución en los tiempos totales. 
En el apartado de diseño, una FPGA está lista en cuanto su diseño incial esté finalizado
y testeado, lo cual, de nuevo, ahorra tiempo. Finalmente, para la aceleración, las FPGA, 
mediante aceleraciones de carga y descarga de información, aumentan el 
rendimiento global del sistema. 

Normalmente la programación de los FPGA se realiza en lenguajes de programación de 
bajo nivel llamados Verilog o VHDL. Ambos son similares y sirven para 'describir' cómo
la FPGA debe manejar el hardware del msimo. Esto se desarrollará con mayor profundidad
y detalle en el apartado software (2.2).

Para el desarrollo de este trabajo se ha elegido como FPGA la Red Pitaya proporcionada 
por Stemlab. Esta FPGA se describe a continuación.

\subsection{Red Pitaya}
[HABLAR A FULL DE LA PITAYA CON SUS CARACTERÍSTICAS Y ESO ADEMÁS DE LAS FOTOS]

\section{Plataforma software}
% [Hablar sobre la plataforma sobre la que se ha programado la fpga (vivado) comentando sus 
% características y sus principales funciones y en que aplicaciones es recomendable su uso.
% Hablar sobre que el programa en c que es el que ejecuta las transmisiones y recepciones 
% y es el encargado de poner en
% funcionamiento el enlace además de comprobar los paquetes, etc.]
La elección de la FPGA es una fase muy importante porque será la base del 
desarrollo de todo el proyecto, por lo tanto, 
en este apartado se va a describir el apartado software con el que se ha implementado
y desarrollado el proyecto. Este apartado consta de dos puntos diferenciales que son la 
programación de la FPGA y la programación en C encargada de generar la trama y de 
controlar la recepción de la misma.

La programación de la FPGA se ha realizado en el entorno Vivado. Vivado Design Suite
es un paquete de software producido por Xilinx para la síntesis y análisis de diseños 
HDL, reemplazando a Xilinx ISE con características adicionales para el desarrollo de 
sistemas en un chip y síntesis de alto nivel. 

Vivado se creó en 2012 y es un entorno de diseño integrado (IDE) que ofrece un ámbito
de desarrollo de próxima generación con orientación SoC, centrado en IP y en el sistema,
que se ha creado desde cero para mejorar la productividad en la integración e 
implementación del sistema.

\chapterend{}
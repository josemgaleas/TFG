%%%%%%%%%%%%%%%%%%%%%%%%%%%%%%%%%%%%%%%%%%%%%%%%%%%%%%%%%%%%%%%%%%%
%%% Documento LaTeX 																						%%%
%%%%%%%%%%%%%%%%%%%%%%%%%%%%%%%%%%%%%%%%%%%%%%%%%%%%%%%%%%%%%%%%%%%
% Título:		Capítulo 1
% Autor:  	Ignacio Moreno Doblas
% Fecha:  	2014-02-01, actualizado 2019-11-11
% Versión:	0.5.0
%%%%%%%%%%%%%%%%%%%%%%%%%%%%%%%%%%%%%%%%%%%%%%%%%%%%%%%%%%%%%%%%%%%
% !TEX root = A0.MiTFG.tex
\chapterbegin{Tecnología empleada}
En este apartado se van a explicar las plataformas o elementos, tanto hardware como software, sobre los que se ha desarrollado este proyecto. 
\vspace{1cm}
\label{chp:ManLaTeX}
\minitoc

\section{Plataforma hardware}
El sistema hardware empleado va a ser un dispositivo FPGA. Una matriz de puertas lógicas programables
(FPGA) se define como un dispositivo electróncio programable que contiene bloques de lógica cuya 
interconexión y funcionalidad puede ser configurada en el momento, mediante un lenguaje de descripción especializado. La lógica programable puede 
reproducir desde funciones tan sencillas como las llevadas a cabo por una puerta lógica o un sistema combinacional hasta complejos sistemas en un chip. 

\subsection{Red Pitaya}

\section{Plataforma software}
Hablar sobre la plataforma sobre la que se ha programado la fpga (vivado) comentando sus características y sus 
principales funciones y en que aplicaciones es recomendable su uso.
Hablar sobre que el programa en c que es el que ejecuta las transmisiones y recepciones y es el encargado de poner en
funcionamiento el enlace además de comprobar los paquetes, etc.

\chapterend{}